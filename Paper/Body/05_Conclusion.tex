\section{Discussion and Conclusion}
\label{sec: conclusion}

Our analysis of counterbalanced within-subjects designs shows the potential challenges in using this methodology for causal inference. While this design can offer advantages—such as increased statistical power, more precise individual-level comparisons, and efficient data collection—it also introduces complications that can bias treatment effect estimates if not carefully accounted for. The primary concern stems from carryover effects, which may persist asymmetrically across treatment conditions, and violations of stability over treatment and time, which can confound treatment estimates across sequential periods. 

Through a formal treatment of these issues within the potential outcomes framework, we demonstrate that counterbalancing does not inherently eliminate biases introduced by differential carryover effects. Our findings suggest that, although counterbalancing is often assumed to mitigate such issues, it relies on the strong and unverifiable assumption that carryover effects are symmetric and cancel out. When this assumption does not hold treatment effect estimates may be substantially biased.

Furthermore, issues that arise in \cwsd{} also appear in other sequential experimental designs where the estimand of interest is the ATE. We formalize the notion of \emph{sequential exchangeability}, which applies to alternative experimental designs to \cwsd{} where the assumptions are more plausibly met. However, we caution that despite the improvements, the difficulty of controlling for carryover effects can substantially hinder the estimation of the ATE and is not a perfect substitute for \bsd{}.

To address these concerns, we propose several practical strategies that researchers can employ when implementing counterbalanced within-subjects designs. First, heuristic checks comparing treatment effects across different time periods can serve as an initial diagnostic tool to detect potential violations of key assumptions. Second, incorporating a washout period between conditions can reduce the impact of lingering treatment effects, particularly in studies involving physiological or psychological interventions. Finally, adjusting for covariate changes between periods can help correct for biases arising from time-dependent confounders that may be influenced by prior treatment exposure, assuming that all carryover effects can be explained indirectly via the covariates.

Despite its limitations, counterbalanced within-subjects design can be a valuable tool for experimental research  participant recruitment is constrained. However, researchers should assess whether \cwsd{} is the most appropriate design choice for their specific study context, or whether alternative methodologies—such as between-subjects designs or sequential randomization—might provide more reliable estimates of causal effects.